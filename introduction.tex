\chapter{概述}

\section{项目概述}

本项目是在“龙芯杯”大赛方提供的FPGA实验平台(FPGA为Artix-7 XC7A200T)上实现一个片上系统(SOC)。其中,CPU基于MIPS 32 Rev 1指令集架构,包含指令缓存和数据缓存;外设包含SPI Flash,GPIO,MAC,DDR3,NAND,串口及LCD屏幕;能够启动并使用PMON操作系统、uCore操作系统,能够启动Linux至用户态初始化。功能方面能够通过大赛方提供的功能测试、性能测试、系统测试;性能方面频率达90MHz,每时钟周期指令数为龙芯GS132的27.044倍。

\subsection{设计框架}
EasterMIPS中的CPU采用顺序双发射五级流水线架构,实现了MIPS32 Rev 1的78条指令,18个CP0寄存器,8个中断,9种例外,8项转换检测缓冲区,固定页大小为4KB。
CPU对外的访存通信通过四个接口,分别是Uncached属性指令接口、Cached属性指令接口、Uncached属性数据接口、Cached属性数据接口。四个接口通过AXI3协议,经过AXI Crossbar整合成为一个接口与外设交互。

EasterMIPS实现了指令缓存 (I-Cache)与数据缓存 (D-Cache),响应CPU的取指与访存请求。I-Cache与D-Cache大小均为16K,在连续命中时,能够实现不间断地流水返回数据。I-Cache与D-Cache分别引出一个AXI接口。D-Cache能够缓冲CPU的写请求,并且实现了一个写回缓存 (Victim Cache),兼具了缓存与写回队列的功能。

\subsection{系统概述}
EasterMips完整支持PMON和Ucore启动运行所需要的指令集,并且支持这这两个系统的全部异常处理。可以完全的运行这两个系统,并执行应用程序。

对于Linux,通过添加CFLAGS构建选项限制分支及修改内核代码去除自陷指令,完成了内核部分的的指令支持。同时使用Buildroot工具链完成了Ramdisk的裁剪构建,去除了原ramdisk中不支持的指令。该ramdisk在官方固件上完成了测试,支持简单的性能监控及登录鉴权。但受限于时间,EasterMips目前仅能完内核初始化过程,无法进入用户态。

\subsection{外设概述}
EasterMips支持SPI Flash板载引导芯片, GPIO, MAC网络控制器, DDR3内存控制器,NAND Flash板载储存, Urat串口通信(RS232), LCD显示。其中LCD屏幕在硬件系统初始化时会显示RGB565格式的16位图像,在系统中经由CONFREG进行显示控制。